\documentclass{article}
\usepackage{graphicx}
\usepackage{amsmath}
\usepackage{hyperref}
\input defs.tex

\title{Project Report}
\author{Jesse Lu}
\begin{document}
\maketitle
\tableofcontents

\section{Main result: Quasi pareto-optimal portfolios}

    The main result of this project is the production of 
        eight portfolio groups with reduced turnover.
    What makes each portfolio group unique is the number of signals
        used to generate the portfolios in each group;
        for this project this was varied from 1 to 128 signals.
    Furthermore, within a portfolio group, 
        each portfolio is distinguished by its emphasis on maximizing
        either the annualized return, or information ratio.
    These results are presented in figure~\ref{result},
        as well as tables~\ref{table:ret}-\ref{table:tvr}.

    Surprisingly, these portfolios can exhibit large returns (up to 7.1)
        and large information ratios (up to 10.2)
        while keeping the turnover well below 0.2 in all cases.
    The method used to generate these portfolios is presented in
        the remainder of this report.

    \begin{figure}[t]
        \centerline{\includegraphics[width=1.8\textwidth]{result.png}}
        \caption{The main result of the project: 
                quasi pareto-optimal portfolios with varying numbers of signals,
                where even small signal numbers can produce large returns
                with large information ratios.
            Here, each point represents a portfolio, 
                while lines connect portfolios 
                generated by models of the same number of signals.
            The variety of portfolios for each signal number group
                is due to differentiated weighting between maximizing
                either the annualized return or the information ratio.
            }
        \label{result}
    \end{figure}
    \clearpage

    \begin{table}[t]\centering
        \begin{tabular}{l c c c c c c c c}
        $n_\text{signals}$ & 1 & 2 & 4 & 8 & 16 & 32 & 64 & 128  \\ \hline
        & 4.519 & 4.548 & 4.591 & 4.819 & 5.501 & 6.271 & 6.848 & 7.182  \\
        & 4.075 & 3.574 & 3.500 & 3.922 & 4.876 & 5.767 & 6.406 & 6.768  \\
        & 1.187 & 2.290 & 2.497 & 3.138 & 4.339 & 5.263 & 5.909 & 6.275  \\
        & 1.241 & 1.824 & 2.127 & 2.869 & 4.086 & 5.020 & 5.654 & 6.011  \\
        & 1.252 & 1.749 & 2.051 & 2.794 & 4.018 & 4.952 & 5.583 & 5.939 
        \end{tabular}
        \caption{Annualized returns of portfolios in figure~\ref{result}.}
        \label{table:ret}
    \end{table}

    \begin{table}[t]\centering
        \begin{tabular}{l c c c c c c c c}
        $n_\text{signals}$ & 1 & 2 & 4 & 8 & 16 & 32 & 64 & 128  \\ \hline
        & 0.348 & 0.375 & 0.425 & 0.513 & 0.641 & 0.739 & 0.801 & 0.837  \\
        & 0.533 & 0.661 & 0.848 & 1.149 & 1.500 & 1.864 & 2.054 & 2.161  \\
        & 0.421 & 0.679 & 1.083 & 1.691 & 2.664 & 3.968 & 4.650 & 4.965  \\
        & 0.416 & 0.655 & 0.999 & 1.759 & 3.466 & 6.304 & 8.300 & 8.933  \\
        & 0.413 & 0.654 & 0.980 & 1.756 & 3.656 & 6.972 & 9.504 & 10.200 
        \end{tabular}
        \caption{Information ratios of portfolios in figure~\ref{result}.}
        \label{table:ir}
    \end{table}


    \begin{table}[t]\centering
        \begin{tabular}{l c c c c c c c c}
        $n_\text{signals}$ & 1 & 2 & 4 & 8 & 16 & 32 & 64 & 128  \\ \hline
        & 0.163 & 0.159 & 0.151 & 0.142 & 0.148 & 0.163 & 0.175 & 0.183  \\
        & 0.131 & 0.123 & 0.116 & 0.114 & 0.130 & 0.153 & 0.170 & 0.179  \\
        & 0.061 & 0.084 & 0.082 & 0.095 & 0.122 & 0.149 & 0.168 & 0.178  \\
        & 0.058 & 0.073 & 0.071 & 0.090 & 0.120 & 0.148 & 0.168 & 0.177  \\
        & 0.058 & 0.071 & 0.069 & 0.089 & 0.119 & 0.148 & 0.167 & 0.177 
        \end{tabular}
        \caption{Turnover of portfolios in figure~\ref{result}.}
        \label{table:tvr}
    \end{table}

 \clearpage


\section{Portfolio-creation strategy}

    We use the following two-part strategy for portfolio creation:
    \begin{enumerate}
        \I  Obtain a near-optimal forward-biased portfolio, which is then used to
        \I  compute a reduced-order, backward-biased model which is used
            to obtain a legitimate backward-biased portfolio.
    \end{enumerate}
            



\section{Finding the near-optimal forward-biased portfolio}

\subsection{Problem statement}
    The portfolio optimization problem can be simplified 
        by substituting expressions for the relevant terms in the following way
        (implemented in \verb+simulate_X.m+):
    \EE{}{
        f_{ir}(x) &= \frac{c^T x}{\std(Ax)} \\
        f_{ret}(x) &= a \frac{c^T x}{\norm{x}_1} \\
        f_{tvr}(x) &= \mean\frac{C|Bx|}{C|x|} \\
        f_{t}(x) &= \frac{\norm{Bx}_1}{\norm{x}_1}} 
    where for an $m$ stocks by $n$ days portfolio,
        \BI $x \in \reals^{mn}$ is a vector representing the portfolio,
        \I  $c$, $a$, $A$, $B$ and $C$ are appropriate vector, scalar, 
            or matrix quantities (see \verb+simulation_matrices.m+), and
        \I  $f_{ir}, f_{ret}, f_{tvr}, f_{t} \in \reals^{mn} \to \reals$
            where $f_{t}$ is an approximation of the turnover function, 
            $f_{tvr}$. \EI

    The problem we want to solve is then
    \EE{original problem}{
        \minimize& - f_{ret}(x) - \hat{\mu}f_{ir}(x) \\
        \subto& f_{tvr}(x) \le 0.2
    }
    That is to say, maximize the return and information ratio
        (with relative proportion affected by coefficient $\hat{\mu}$)
        while keeping the turnover below $0.2$.


    Note that for brevity, I have not included the expressions for 
        $c$, $a$, $A$, $B$ and $C$; 
        however, one can deduce that such quantities must exist.
    For example, notice that $c^T x$ represents the total profit/loss;
        we know this expression must be valid for some $c$ 
        since the expression for the total profit/loss is linear.
    Lastly, notice that the portfolio is represented as a vector ($x$),
        rather than a matrix, in this section for notational convenience.

\subsection{Problem simplification}
    We now simplify \eq{original problem} in order 
        to make it computationally tractable.

    We first note that \eq{original problem} is homogenous,
        meaning that it's solution depends only on the relative values of $x$.
    Therefore we can split the terms of $f_{tvr}(x)$ in the following way
        without loss of generality.
    \EE{}{
        \minimize& - f_{ret}(x) - \hat{\mu}f_{ir}(x) \\
        \subto& \norm{Bx}_1 \le 0.2 \\
            & \norm{x}_1 = 1
    }

    Turning our attention to the objective function we note that
    \E{}{
        f_{ret}(x) + \hat{\mu}f_{ir}(x) = 
            \frac{c^T x}{\std(Ax)} + \hat{\mu} a c^T x}
        since $\norm{x}_1 = 1$.
    This objective can be thought of as 
        weighting two terms by $\hat{\mu}$;
        and we can alternatively weight the following two terms
    \E{}{
        \frac{c^T x}{\std(Ax)} + \hat{\mu} a c^T x \to 
            (1/\mu)c^T x - \std(Ax)}
        while still obtaining the full range of pareto-optimal points;
        although the connection between $\mu$ and $\hat{\mu}$
        is not immediately clear.

    This now gives us
    \EE{}{
        \minimize& (-1/\mu)c^T x + \std(Ax) \\
        \subto& \norm{Bx}_1 \le 0.2 \\
            & \norm{x}_1 = 1
    }
        which is still completely equivalent to \eq{original problem}.

    We now implement the following generalization
    \E{}{   \norm{x}_1 = 1 \quad\to\quad \norm{x}_1 \le 1}
        in order to make the problem 
        convex\footnote{see Boyd and Vandenberghe, ``Convex Optimization''}.
    Our reasoning behind this approximation
        is that although $\norm{x}_1$ is no longer 
        strictly confined to the value of 1,
        it most likely will end up with a quantity of 1 anyways
        because the $c^T x$ term in the objective function will always want
        to ``push'' $\norm{x}_1$ to as large a value as possible.

    While this approximation does inject error into our formulation,
        the benefits of convexity (i.e. the ability to efficiently compute
        the global minimum) far outweigh the costs.
    Actually, the cost of this approximation can be made negligible by
        twiddling the value of $0.2$ 
        in the $\norm{Bx}_1 \le 0.2$ constraint 
        (using a simple algorithm like bisection for instance)
        in order to force $f_{tvr}(x) \le 0.2$.
    
    We have now simplified \eq{original problem} into the following convex form
    \EE{}{
        \minimize& (-1/\mu)c^T x + \std(Ax) \\
        \subto& \norm{Bx}_1 \le 0.2 \\
            & \norm{x}_1 \le 1
    }
        which can then be further approximated as 
    \EE{convex problem}{
        \minimize& (-1/\mu)c^T x + \std(Ax) \\
        \subto& \norm{Bx}_2 \le 0.2 \\
            & \norm{x}_2 \le 1
    }
        if desired for stability purposes.

\subsection{Solution method}
    We use the CVX package\footnote{\url{www.cvxr.com/cvx}} to solve 
        \eq{convex problem}, which takes about 5 minutes on a desktop computer
        for the problem size given.

    Note that because the problem in \eq{convex problem} is convex
        the global minimum can deterministically and efficiently be found.
    Furthermore, once found the optimal point can be proven to be globally
        optimal.

    It is from this that we claim that the result of \eq{convex problem}
        would be optimal, if not for the two approximations used earlier.
    However, we also note that such a portfolio would be forward-biased,
        since the entire price history is available 
        to the optimization program.

    Lastly, we note that the use of a simple algorithm, such as bisection,
        can be coupled with this solution method
        in order to yield portfolios with an exact turnover 
        as decided by the user.

\section{Computing the reduced-order model}   

    Although the portfolio obtained 
        via the method presented in the previous section
        would not be valid because of forward-biasing,
        it is extremely useful in building a reduced-order model
        for creating a portfolio that is backward-biased only.
    The full-order version of this model, $M$, is computed via
    \EE{}{
        MY =& X_\text{FB} \\
        M =& X_\text{FB}Y^+}
    where for an $m$ stocks and $n$ days portfolio,
        \BI $M \in \reals^{m \times p}$ is the full-order model matrix,
        \I  $Y \in \reals^{p \times n}$ is the signals matrix 
            which contains the backwards-looking signals 
            (e.g. percent returns for past 5 days)
            which are fed into the model,
        \I  $Y^+ \in \reals^{n \times p}$ is the pseudo-inverse of $Y$,
        \I  $X_\text{FB} \in \reals^{m \times n}$ is the matrix representing 
            the near-optimal portfolio obtained from the previous section.
            Note that we now use a matrix convention for convenience. \EI

    To obtain a reduced $q$-order model, 
        we simply take the $q$ most significant singular components of $M$
        which can be computed via a singular value decomposition of $M$,
    \EE{}{
        M =& U\Sigma V^T = U \text{diag}(\sigma) V^T \\
        M_q =& u_1 \sigma_1 v_1^T + u_2 \sigma_2 v_2^T + \cdots + u_q \sigma_q v_q^T.}
    In this way, the reduced-order modeling matrix, $M_q$,
        has the same shape as $M$, but is only of order $q$.
    Thus, we say that $M_q$ is a model with $q$ signals,
        and when used, creates a $q$-signal portfolio.

\section{Conclusion}
    

\end{document}
